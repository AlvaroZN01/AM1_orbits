% !TeX spellcheck = es_ES

%------------Basic document settings--------------%
\documentclass[12pt, openany]{article}
\usepackage[a4paper,width = 150mm, top=25mm, bottom= 25mm, footskip=35pt]{geometry}
\usepackage[utf8]{inputenc}
\usepackage[spanish]{babel}

%---------Graphics paths and labels------------%
\usepackage{graphicx}
\graphicspath{{Images/}}
\usepackage[labelfont=bf]{caption}
\usepackage{subcaption}
\usepackage{titlesec}

%-Miscellaneous packages for different purposes-%
\usepackage{chngcntr}
\usepackage{setspace}
\usepackage{fancyhdr}
\usepackage{url}
\usepackage{lastpage} 
\usepackage[bottom]{footmisc}
\usepackage{pdflscape}
\usepackage{ragged2e}
\usepackage{enumerate}
\usepackage{enumitem} 
\usepackage{todonotes}
\usepackage{multirow}
\usepackage{bm}
\usepackage{afterpage}
\usepackage{tabularx}
\usepackage{float}

%-------------Math and Physics packages-----------%
\usepackage{amssymb}
\usepackage{amsthm}
\usepackage{amsmath}
\usepackage{amsfonts}
\usepackage{mhchem}
\usepackage{physics}
\usepackage{mathtools}
\usepackage{empheq}

%---------Sets up different colors----------------%

\usepackage{hyperref}
\usepackage{xcolor}
\usepackage{layouts}
\hypersetup{
	colorlinks=true,
	linkcolor=blue,
	filecolor=magenta,      
	urlcolor=blue,
	citecolor=blue,
}
\usepackage{listings}
\usepackage{color} %red, green, blue, yellow, cyan, magenta, black, white
\definecolor{mygreen}{RGB}{28,172,0} % color values Red, Green, Blue
\definecolor{mylilas}{RGB}{170,55,241}
%\usepackage{tabu}

%---------Set specific bibliography--------------%

\usepackage{natbib}
\bibliographystyle{plainnat}
\setcitestyle{authoryear,open={(},close={)}}

%--------Useful for Mathematical Texts-----------%

\newtheorem{theorem}{Theorem}[section]
\newtheorem{corollary}{Corollary}[theorem]
\newtheorem{lemma}[theorem]{Lemma}
\newtheorem{definition}{Definition}[section]
\newtheorem{prop}{Proposition}
\newtheorem{property}{Property}
\theoremstyle{definition}
\newtheorem{remark}{Remark}[section]
\newtheorem{example}{Example}[section]

%--------Definition of divergente operator-------%
\AtBeginDocument{
	\let\div\relax
	\DeclareMathOperator{\div}{div}
}
\DeclareMathOperator{\ran}{ran}
\newcommand*\df{\mathop{}\!\mathrm{d}}
\newcommand*\Df[1]{\mathop{}\!\mathrm{d^#1}}

%---------Header and footer definitions----------%

% Define Header
\fancyhf{}
\fancyhead[L]{\hspace{0.3cm}{\leftmark}} 

% Define Footer
\fancyfoot[L]{\hspace{0.3cm} Álvaro Zurdo Navajo}
\fancyfoot[C]{\thepage}
\fancyfoot[R]{Trabajo 1 - MCPS \hspace{0.3cm}} 

%Define width of horizontal line in the header
\renewcommand{\headrulewidth}{0.4pt}
\setlength{\headsep}{0.2in}

%Define width of horizontal line in the footer
\renewcommand{\footrulewidth}{0.4pt}
\setlength{\headheight}{25pt}

%--Establish when figures and tables begin to count--%

\counterwithin{figure}{section}
\counterwithin{table}{section}
\numberwithin{equation}{section}

%--Pasar algunas palabras a español--%
%\renewcommand{\listtablename}{Índice de tablas}
\addto\captionsspanish{%
  \renewcommand{\listtablename}{Lista de tablas}%
  \renewcommand{\listfigurename}{Lista de figuras}%
  \renewcommand{\tablename}{Tabla}%
  \renewcommand{\lstlistingname}{Código}
}


%----------Use of nomenclature section-----------%
\usepackage{enumitem}
\usepackage{nomencl}
\usepackage{etoolbox}
\usepackage{acronym}
%\renewcommand\nomgroup[1]{%
%	\item[\bfseries
%	\ifstrequal{#1}{A}{Chapter 2}{%
%	\ifstrequal{#1}{B}{Chapter 3}{%
%	\ifstrequal{#1}{C}{Chapter 4}{}}}%
%	]}
\renewcommand\nomgroup[1]{%
	\item[\bfseries
	\ifstrequal{#1}{A}{Notation}{%
	\ifstrequal{#1}{B}{Symbols}{}}%
	]}
\makenomenclature

%----------Code formating-----------%

\definecolor{codegreen}{rgb}{0,0.6,0}
\definecolor{codegray}{rgb}{0.5,0.5,0.5}
\definecolor{codepurple}{rgb}{0.58,0,0.82}
\definecolor{backcolour}{rgb}{0.95,0.95,0.92}

\lstdefinestyle{mystyle}{
    backgroundcolor=\color{backcolour},
    commentstyle=\color{codegreen},
    keywordstyle=\color{magenta},
    numberstyle=\tiny\color{codegray},
    stringstyle=\color{codepurple},
    basicstyle=\footnotesize,
    breakatwhitespace=false,
    breaklines=true,
    captionpos=b,
    keepspaces=true,
    numbers=left,
    numbersep=5pt,
    showspaces=false,
    showstringspaces=false,
    showtabs=false,
    tabsize=2
}

\lstset{style=mystyle}

\begin{document}

\pagenumbering{roman}
\onehalfspacing

\begin{center}
\thispagestyle{empty}


\begin{figure}
	\begin{subfigure}{0.5\textwidth}
		\includegraphics[width=\linewidth]{Images/Portada/Logo_UPM.png}\\
	\end{subfigure}
	\begin{subfigure}{0.5\textwidth}
		\includegraphics[width=\linewidth]{Images/Portada/Logo_ETSIAE.png}\\
	\end{subfigure}
%	\begin{subfigure}{0.32\textwidth}
%		\includegraphics[width=\linewidth]{Images/Portada/Logo_MUSE.png}\\
%	\end{subfigure}
\end{figure}

\begin{figure}
\centering
\includegraphics[width=0.75\linewidth]{Images/Portada/Logo_MUSE.png}
\end{figure}

\vspace*{3cm} % Este comando no funciona, comprobar por qué
{\huge \textbf{Hito 1}}\\
\vspace{0.75cm}
\textbf{Ampliación de matemáticas} \\
\begin{center}
    \large{\textbf{\today}} \\
\end{center}

\vspace{5cm} % 6.5

\begin{center}
        \large{Álvaro Zurdo Navajo}
\end{center}

\end{center}
\clearpage

\doublespacing	
\tableofcontents
\pagestyle{plain}
\clearpage

%\addcontentsline{toc}{section}{\listfigurename}
%\listoffigures
%\clearpage
%\thispagestyle{plain}
%
%\addcontentsline{toc}{section}{\listtablename}
%\listoftables
%\clearpage

\pagestyle{fancy}
\pagenumbering{arabic}
\setcounter{page}{1}	
\onehalfspacing		
\section{Introducción}
	\label{chap:intro}
	% !TeX spellcheck = es_ES

En el segundo hito se pretende realizar la integración de ecuaciones llamando a funciones, cuyos nombres son los esquemas numéricos que contienen. Además, se va a añadir la integración mediante el Método de Euler Inverso. \\

Además de estas funciones para cada uno de los esquemas temporales, se pide crear una función para resolver problemas de Cauchy en general. Esta función deberá tener como inputs el esquema temporal, las condiciones iniciales y la función a integrar, como mínimo. También se incluye como input el vector de tiempo, que contiene los instantes temporales en los que se obtendrán los resultados. \\

También se pide almacenar la función a integrar (la ecuación de las órbitas de Kepler) en una función aparte, permitiendo así que pueda ser introducida como input en la función de Cauchy. \\

Finalmente se pide integrar la ecuación empleando los cuatro esquemas temporales, modificando el incremento de tiempo y comparando los resultados. 
\clearpage
	
\section{Desarrollo}
	\label{chap:1}
	\input{Chapters/Chapter1}
\clearpage

\section{Código}
	\label{chap:2}
	% !TeX spellcheck = es_ES

\subsection{Temporal\_schemes}

\begin{lstlisting}[language=Python, caption=Código de la función Temporal\_schemes, label=Hito_2_Temporal_schemes_code]
from scipy import optimize

def Euler(U0, t0, tf, f):
    return U0 + (tf - t0) * f(t0, U0)

def Crank_Nicolson(U0, t0, tf, f):
    def Residual(x):
        return x - U0 - (tf - t0)/2 * (f(t0, U0) + f(tf, x))
    return optimize.newton(func = Residual, x0 = U0)

def RK4(U0, t0, tf, f):
    dt = tf - t0
    k1 = f(t0, U0)
    k2 = f(t0 + dt/2, U0 + k1*dt/2)
    k3 = f(t0 + dt/2, U0 + k2*dt/2)
    k4 = f(t0 + dt, U0 + k3*dt)
    return U0 + dt/6 * (k1 + 2*k2 + 2*k3 + k4)

def Inverse_Euler(U0, t0, tf, f):
    def Residual(x):
        return x - U0 - (t0 - tf) * f(tf, x)
    return optimize.newton(func = Residual, x0 = U0)

\end{lstlisting}

\subsection{Cauchy\_problem}

\begin{lstlisting}[language=Python, caption=Código de la función Cauchy\_problem, label=Hito_2_Cauchy_problem_code]
from numpy import array, zeros

def Cauchy(t, temporal_scheme, f, U0):
    U = array (zeros((len(U0),len(t))))
    U[:,0] = U0
    for ii in range(0, len(t) - 1):
        U[:,ii+1] = temporal_scheme(U[:,ii], t[ii], t[ii+1], f)
    return U

\end{lstlisting}

\subsection{Función principal (Hito\_2)}

\begin{lstlisting}[language=Python, caption=Código principal del Hito 2, label=Hito_2_main_code]
from numpy import arange, array
import matplotlib.pyplot as plt
from ODEs.Cauchy_problem import Cauchy
from ODEs.Temporal_schemes import Euler, Inverse_Euler, RK4, Crank_Nicolson

# Function to be integrated
def F_Kepler(t, U):
    
    x, y, vx, vy = U[0], U[1], U[2], U[3]
    mr = (x**2 + y**2)**1.5
    return array([vx, vy, -x/mr, -y/mr])

# Integration parameters definition
dt = 0.001
N = 10000
t0 = 0
t = arange(t0, N * dt, dt)
U0 = [1, 0, 0, 1]

# Cauchy problem solver
U = Cauchy(t, Crank_Nicolson, F_Kepler, U0)

# Plot results
plt.axis('equal')
plt.plot(U[0,:],U[1,:])
plt.show()

\end{lstlisting}












\clearpage

\section{Resultados}
	\label{chap:resultados}
	\input{Chapters/Resultados}

\end{document}}
		