% !TeX spellcheck = en_US

%------------Basic document settings--------------%
\documentclass[12pt, twoside, openright]{report}
\usepackage[a4paper,width = 150mm, top=25mm, bottom= 25mm, bindingoffset = 14mm, footskip=35pt]{geometry}
\usepackage[utf8]{inputenc}
\usepackage[spanish]{babel}

%---------Graphics paths and labels------------%
\usepackage{graphicx}
\graphicspath{{Images/}}
\usepackage[labelfont=bf]{caption}
\usepackage{subcaption}
\usepackage{titlesec}

%-Miscellaneous packages for different purposes-%
\usepackage{chngcntr}
\usepackage{setspace}
\usepackage{fancyhdr}
\usepackage{url}
\usepackage{lastpage} 
\usepackage[bottom]{footmisc}
\usepackage{pdflscape}
\usepackage{ragged2e}
\usepackage{enumerate}
\usepackage{enumitem} 
\usepackage{todonotes}
\usepackage{multirow}
\usepackage{bm}
\usepackage{floatpag}
\usepackage{afterpage}

%-------------Math and Physics packages-----------%
\usepackage{amssymb}
\usepackage{amsthm}
\usepackage{amsmath}
\usepackage{amsfonts}
\usepackage{mhchem}
\usepackage{physics}
\usepackage{mathtools}
\usepackage{empheq}

%---------Sets up different colors----------------%

\usepackage{hyperref}
\usepackage{xcolor}
\usepackage{layouts}
\hypersetup{
	colorlinks=true,
	linkcolor=blue,
	filecolor=magenta,      
	urlcolor=blue,
	citecolor=blue,
}
\usepackage{listings}
\usepackage{color} %red, green, blue, yellow, cyan, magenta, black, white
\definecolor{mygreen}{RGB}{28,172,0} % color values Red, Green, Blue
\definecolor{mylilas}{RGB}{170,55,241}
%\usepackage{tabu}

%---------Set specific bibliography--------------%

\usepackage{natbib}
\bibliographystyle{plainnat}
\setcitestyle{authoryear,open={(},close={)}}

%--------Useful for Mathematical Texts-----------%

\newtheorem{theorem}{Theorem}[section]
\newtheorem{corollary}{Corollary}[theorem]
\newtheorem{lemma}[theorem]{Lemma}
\newtheorem{definition}{Definition}[section]
\newtheorem{prop}{Proposition}
\newtheorem{property}{Property}
\theoremstyle{definition}
\newtheorem{remark}{Remark}[section]
\newtheorem{example}{Example}[section]

%--------Definition of divergente operator-------%
\AtBeginDocument{
	\let\div\relax
	\DeclareMathOperator{\div}{div}
}
\DeclareMathOperator{\ran}{ran}
\newcommand*\df{\mathop{}\!\mathrm{d}}
\newcommand*\Df[1]{\mathop{}\!\mathrm{d^#1}}

%---------Header and footer definitions----------%

% Define Header
\fancyhead[LE]{\hspace{0.3cm}{\leftmark}} 
\fancyhead[RO]{{\rightmark}\hspace{0.3cm}} 
\fancyhead[RE]{} 
\fancyhead[LO]{} 

% Define Footer
\lfoot{\hspace{0.3cm} Álvaro Zurdo Navajo}
\cfoot{\thepage}
\rfoot{Cambiar por asignatura y trabajo \hspace{0.3cm}} 

%Define width of horizontal line in the header
\renewcommand{\headrulewidth}{0.4pt}
\setlength{\headsep}{0.2in}

%Define width of horizontal line in the footer
\renewcommand{\footrulewidth}{0.4pt}
\setlength{\headheight}{25pt}

%--Establish when figures and tables begin to count--%

\counterwithin{figure}{chapter}
\counterwithin{table}{chapter}
\numberwithin{equation}{chapter}

%--Pasar algunas palabras a español--%
%\renewcommand{\listtablename}{Índice de tablas}
\addto\captionsspanish{%
  \renewcommand{\listtablename}{Índice de tablas}%
}

%----------Use of nomenclature section-----------%
\usepackage{enumitem}
\usepackage{nomencl}
\usepackage{etoolbox}
\usepackage{acronym}
%\renewcommand\nomgroup[1]{%
%	\item[\bfseries
%	\ifstrequal{#1}{A}{Chapter 2}{%
%	\ifstrequal{#1}{B}{Chapter 3}{%
%	\ifstrequal{#1}{C}{Chapter 4}{}}}%
%	]}
\renewcommand\nomgroup[1]{%
	\item[\bfseries
	\ifstrequal{#1}{A}{Notation}{%
	\ifstrequal{#1}{B}{Symbols}{}}%
	]}
\makenomenclature
\begin{document}

\pagenumbering{roman}
\onehalfspacing

\begin{center}
\thispagestyle{empty}


\begin{figure}
	\begin{subfigure}{0.5\textwidth}
		\includegraphics[width=\linewidth]{Images/Portada/Logo_UPM.png}\\
	\end{subfigure}
	\begin{subfigure}{0.5\textwidth}
		\includegraphics[width=\linewidth]{Images/Portada/Logo_ETSIAE.png}\\
	\end{subfigure}
%	\begin{subfigure}{0.32\textwidth}
%		\includegraphics[width=\linewidth]{Images/Portada/Logo_MUSE.png}\\
%	\end{subfigure}
\end{figure}

\begin{figure}
\centering
\includegraphics[width=0.75\linewidth]{Images/Portada/Logo_MUSE.png}
\end{figure}

\vspace*{3cm} % Este comando no funciona, comprobar por qué
{\huge \textbf{Hito 1}}\\
\vspace{0.75cm}
\textbf{Ampliación de matemáticas} \\
\begin{center}
    \large{\textbf{\today}} \\
\end{center}

\vspace{5cm} % 6.5

\begin{center}
        \large{Álvaro Zurdo Navajo}
\end{center}

\end{center}

%\chapter*{Abstract}
%\addcontentsline{toc}{chapter}{Abstract}
%
%The objective of this thesis is to asses the performance of a Kalman filter when it is applied to an airborne-launched orbital launcher. Specifically, the study focuses on the captured flight phase of the launching sequence, testing the effectiveness of the filter during an aborted launch maneuver, in which the launcher remains attached to the airplane. To conduct the analysis, a simulated flight profile is created and implemented into a simulator, which provides simulated measurements from the launcher's IMU and GNSS sensors. The Kalman filter is then tested using those measurements, estimating the launcher's position, velocity and attitude. The study aims to evaluate the algorithm's accuracy at estimating the launcher's state variables during this flight profile, as well as the effects of measurement errors from the sensors and initial conditions variations on the accuracy of the estimation. Extensive simulations and analysis are performed using Monte-Carlo methods to asses the effectiveness of the Kalman filter, providing valuable insights into its behavior during the captured phase of the launcher's flight. \\
%
%\textbf{Key words:} Kalman filter, navigation, IMU, GNSS, sensor error modeling, flight mechanics, ODE, quaternions, Monte-Carlo, flat Earth hypothesis correction. 

\doublespacing	
\tableofcontents
\pagestyle{plain}

\addcontentsline{toc}{chapter}{\listfigurename}
\listoffigures
\cleardoublepage
\thispagestyle{plain}

\addcontentsline{toc}{chapter}{List of Tables}
\listoftables
\cleardoublepage

%\onehalfspacing
%\addcontentsline{toc}{chapter}{Nomenclature}
%\printnomenclature

%\nomenclature[A]{$a$, $\beta$, $C$}{Scalars}

%\nomenclature[B]{$T$}{Thrust}

\clearpage

%\textbf{\Large
%	{Acronyms}}
%\begin{acronym}[MEMS]
%	\acro{GNC}{Guidance, Navigation and Control}
%\end{acronym}

\cleardoublepage

%\textbf{\Large{Symbols}} \\
%\cleardoublepage

\pagestyle{fancy}
\pagenumbering{arabic}
\setcounter{page}{1}	
\onehalfspacing		
\chapter{Introducción}
	\label{chap:intro}
	% !TeX spellcheck = es_ES

En el segundo hito se pretende realizar la integración de ecuaciones llamando a funciones, cuyos nombres son los esquemas numéricos que contienen. Además, se va a añadir la integración mediante el Método de Euler Inverso. \\

Además de estas funciones para cada uno de los esquemas temporales, se pide crear una función para resolver problemas de Cauchy en general. Esta función deberá tener como inputs el esquema temporal, las condiciones iniciales y la función a integrar, como mínimo. También se incluye como input el vector de tiempo, que contiene los instantes temporales en los que se obtendrán los resultados. \\

También se pide almacenar la función a integrar (la ecuación de las órbitas de Kepler) en una función aparte, permitiendo así que pueda ser introducida como input en la función de Cauchy. \\

Finalmente se pide integrar la ecuación empleando los cuatro esquemas temporales, modificando el incremento de tiempo y comparando los resultados. 
	
\chapter{Capítulo 1}
	\label{chap:2}
	% !TeX spellcheck = es_ES

\subsection{Temporal\_schemes}

\begin{lstlisting}[language=Python, caption=Código de la función Temporal\_schemes, label=Hito_2_Temporal_schemes_code]
from scipy import optimize

def Euler(U0, t0, tf, f):
    return U0 + (tf - t0) * f(t0, U0)

def Crank_Nicolson(U0, t0, tf, f):
    def Residual(x):
        return x - U0 - (tf - t0)/2 * (f(t0, U0) + f(tf, x))
    return optimize.newton(func = Residual, x0 = U0)

def RK4(U0, t0, tf, f):
    dt = tf - t0
    k1 = f(t0, U0)
    k2 = f(t0 + dt/2, U0 + k1*dt/2)
    k3 = f(t0 + dt/2, U0 + k2*dt/2)
    k4 = f(t0 + dt, U0 + k3*dt)
    return U0 + dt/6 * (k1 + 2*k2 + 2*k3 + k4)

def Inverse_Euler(U0, t0, tf, f):
    def Residual(x):
        return x - U0 - (t0 - tf) * f(tf, x)
    return optimize.newton(func = Residual, x0 = U0)

\end{lstlisting}

\subsection{Cauchy\_problem}

\begin{lstlisting}[language=Python, caption=Código de la función Cauchy\_problem, label=Hito_2_Cauchy_problem_code]
from numpy import array, zeros

def Cauchy(t, temporal_scheme, f, U0):
    U = array (zeros((len(U0),len(t))))
    U[:,0] = U0
    for ii in range(0, len(t) - 1):
        U[:,ii+1] = temporal_scheme(U[:,ii], t[ii], t[ii+1], f)
    return U

\end{lstlisting}

\subsection{Función principal (Hito\_2)}

\begin{lstlisting}[language=Python, caption=Código principal del Hito 2, label=Hito_2_main_code]
from numpy import arange, array
import matplotlib.pyplot as plt
from ODEs.Cauchy_problem import Cauchy
from ODEs.Temporal_schemes import Euler, Inverse_Euler, RK4, Crank_Nicolson

# Function to be integrated
def F_Kepler(t, U):
    
    x, y, vx, vy = U[0], U[1], U[2], U[3]
    mr = (x**2 + y**2)**1.5
    return array([vx, vy, -x/mr, -y/mr])

# Integration parameters definition
dt = 0.001
N = 10000
t0 = 0
t = arange(t0, N * dt, dt)
U0 = [1, 0, 0, 1]

# Cauchy problem solver
U = Cauchy(t, Crank_Nicolson, F_Kepler, U0)

# Plot results
plt.axis('equal')
plt.plot(U[0,:],U[1,:])
plt.show()

\end{lstlisting}












	
\chapter{Capítulo 2}
	\label{chap:3}
	\input{Chapters/Chapter3}

\chapter{Capítulo 3}
	\label{chap:4}
	\input{Chapters/Chapter4}

\chapter{Conclusión}
	\label{chap:conclusiones}
	\input{Chapters/Conclusion}
	
\appendix
\chapter{Ejemplo de anexo (Sistemas de referencia)}
	\label{chap:ap_ref_sys}
	The reference systems used throughout the project are explained below: 

\begin{itemize}

\item Body frame (B): 

These axes are continuously fixed to the aircraft, with their origin situated in the airplane's center of mass. The x-axis is contained in the airplane's symmetry plane and is pointing towards the front of the aircraft. The z-axis is also contained in the aircraft's symmetry plane and is pointing towards the bottom of the aircraft, being perpendicular to the x-axis. Meanwhile, the y-axis forms a right-handed trihedron with the other two axis, and points towards the right semi-span. This frame is represented in figure \ref{fig:B_frame}.

\begin{figure}
\centering
\includegraphics[scale=0.4]{Images/B_frame.jpg}
\caption{Representation of the body frame of reference}
\label{fig:B_frame}
\end{figure}

\item Wind frame (W):

These axes are also centered in the aircraft's center of mass. The x-axis is parallel to the aerodynamic velocity vector of the airplane and has the same direction. The z-axis is contained in the aircraft's symmetry plane and is pointing towards the bottom of the aircraft, being perpendicular to the x-axis. The y-axis forms a right-handed thriedron with the last two axis. This frame is represented in figure \ref{fig:W_frame}.

\begin{figure}
\centering
\includegraphics[scale=0.4]{Images/W_frame.jpg}
\caption{Representation of the wind frame of reference}
\label{fig:W_frame}
\end{figure}

\newpage

\item Thrust frame:

These axes are defined the same way as the wind axes, but the x-axis is parallel to the thrust vector and points towards the same direction. This frame is represented in figure \ref{fig:T_frame}.

\begin{figure}
\centering
\includegraphics[scale=0.4]{Images/T_frame.jpg}
\caption{Representation of the thrust frame of reference}
\label{fig:T_frame}
\end{figure}

\item Flat Earth frame (FE):

These axes have their origin in the surface point below the initial position of the aircraft. The x axis points towards the initial direction of the aircraft projected on the horizontal plane. The z axis is perpendicular to that horizontal plane and points towards the center of the Earth (below the horizontal plane, considering that a flat Earth would not have a center). The y-axis forms a right-handed thriedron with the last two axis. This way, the position of the aircraft in this frame at t=0 seconds would be $[0, 0, -h_0]$. This frame is represented in figure \ref{fig:FE_frame}.

\begin{figure}
\centering
\includegraphics[scale=0.4]{Images/FE_frame.jpg}
\caption{Representation of the flat Earth frame of reference}
\label{fig:FE_frame}
\end{figure}

\item Local Horizon frame (LH):

These axes have their origin at the same point as the Flat Earth axes and share the same z-axis, but the x-axis is oriented towards the north of the Earth and the y-axis points towards the East. This frame is represented in figure \ref{fig:LH_frame}.

\begin{figure}
\centering
\includegraphics[scale=0.4]{Images/LH_frame.jpg}
\caption{Representation of the local horizon frame of reference}
\label{fig:LH_frame}
\end{figure} 

\newpage

\item Earth Centered Fixed Intermediate frame (ECFI): 

These axis are centered in the Earth's ellipsoid center and rotate with the Earth. The z-axis is parallel to the Earth's rotation axis. The x-axis is always contained in the same plane as the z-axis and the initial position, it is perpendicular to the z-axis and points towards that initial point. The y-axis forms a right-handed thriedron with the last two axis. This frame is represented in figure \ref{fig:ECFI_frame}.

\begin{figure}
\centering
\includegraphics[scale=0.4]{Images/ECFI_frame.jpg}
\caption{Representation of the Earth centered fixed intermediate frame of reference}
\label{fig:ECFI_frame}
\end{figure}

\item Earth Centered Fixed frame (ECF): 

These axes are just like the ECFI ones, but the x-axis always points towards the Greenwich Meridian (latitude 0) instead of pointing towards the initial position of the plane. These axes rotate with the Earth, just like the ECFI axes. This frame is represented in figure \ref{fig:ECF_frame}.

\begin{figure}
\centering
\includegraphics[scale=0.4]{Images/ECF_frame.jpg}
\caption{Representation of the Earth centered fixed frame of reference}
\label{fig:ECF_frame}
\end{figure}

\item Earth Centered Inertial frame (ECI): 

These axes are similar to the ECF axes, but are not fixed to the Earth. The x-axis points towards the Earth's mean equinox, which is the intersection between the equatorial and ecliptic planes, while the z-axis is parallel to the Earth's rotation axis and points towards the north, and the y-axis forms a right handed thriedron with the other axis. This frame is represented in figure \ref{fig:ECI_frame}. 

\begin{figure}
\centering
\includegraphics[scale=0.4]{Images/ECI_frame.jpg}
\caption{Representation of the Earth centered inertial frame of reference}
\label{fig:ECI_frame}
\end{figure}

\end{itemize}

	
\nocite{*}
\bibliography{biblio}


\end{document}}
		