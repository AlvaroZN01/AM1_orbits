% !TeX spellcheck = es_ES

En el segundo hito se pretende realizar la integración de ecuaciones llamando a funciones, cuyos nombres son los esquemas numéricos que contienen. Además, se va a añadir la integración mediante el Método de Euler Inverso. \\

Además de estas funciones para cada uno de los esquemas temporales, se pide crear una función para resolver problemas de Cauchy en general. Esta función deberá tener como inputs el esquema temporal, las condiciones iniciales y la función a integrar, como mínimo. También se incluye como input el vector de tiempo, que contiene los instantes temporales en los que se obtendrán los resultados. \\

También se pide almacenar la función a integrar (la ecuación de las órbitas de Kepler) en una función aparte, permitiendo así que pueda ser introducida como input en la función de Cauchy. \\

Finalmente se pide integrar la ecuación empleando los cuatro esquemas temporales, modificando el incremento de tiempo y comparando los resultados. 